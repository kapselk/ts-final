\documentclass[11pt,a4paper]{article}

\usepackage{polski}
\usepackage[utf8]{inputenc}

\usepackage[left=2cm, right=2cm, top=1cm, bottom=2cm]{geometry}

\usepackage{graphicx}
\usepackage{mathtools}
\usepackage{amssymb}
\usepackage{amsthm}
\usepackage{thmtools}
\usepackage{xcolor}
\usepackage{nameref}
\usepackage{babel}
\usepackage{hyperref}
\usepackage{enumitem}
\usepackage{amsmath, amsthm}

\usepackage{listings}
\usepackage{xcolor}
\lstdefinestyle{Rstyle}{
  language=R,
  basicstyle=\ttfamily\small,
  keywordstyle=\color{blue},
  commentstyle=\color{gray}\itshape,
  stringstyle=\color{purple},
  numberstyle=\tiny\color{gray},
  numbers=left,
  stepnumber=1,
  showstringspaces=false,
  breaklines=true,
  frame=single,
  captionpos=b
}

% Definicje nowych środowisk
\newtheorem{theorem}{Theorem}
% \newtheorem{lemma}{Lemat}
% \newtheorem{fact}{Fakt}
% \newtheorem{corollary}{Wniosek}
% \newtheorem{exercise}{Zadanie}[section]
% \newtheorem{solution}{Rozwiązanie}
% \newtheorem{definition}{Definicja}
% \newtheorem{namedtheorem}{Twierdzenie}[section]
% \newtheorem{example}{Przykład}

\newcommand{\PP}{\mathbb P}
\newcommand{\EE}{\mathbb E}
\newcommand{\Z}{\mathbb Z}
\newcommand{\R}{\mathbb R}
\newcommand{\Poi}{\mathrm{Poiss}}
\newcommand{\F}{\mathcal F}
\newcommand{\tn}{{\tau \wedge n}}
\newcommand{\var}{\mathrm{Var}}
\newcommand{\cov}{\mathrm{Cov}}
\newcommand{\N}{\mathbb N}

\begin{document}

    \title{Time Series \\ Examinations Solutions}
    \author{Kacper Omieliańczyk (459458)}
    \date{}

    \maketitle
    \subsubsection*{Task 1}
    We write the model as \[ X_t = -0.25 X_{t-1} + Z_t \]
    Since $\{Z_t\}$ is white noise with $\var(Z_t) = 1$, it is well-known
    \[ \gamma(k) = \cov(X_t, X_{t-k}) = \frac{\sigma_Z^2}{1-\phi^2} \phi^{|k|} \]
    where $\phi = 0.25, \sigma_Z^2 = 1$. Thus
    \[ 1 - \phi^2 = \frac{15}{16}, \quad \gamma(k) = \frac{16}{15} (-0.25)^{|k|} \]
    In particular
    \[ \gamma(0) = \frac{16}{15}, \quad \gamma(1) = -\frac4{15}, \quad \gamma(2) = \frac1{15} \]
    
    We seek $(a_1, a_2)$ minimizing
    \[ \mathrm{MSE} = \EE\left[ (X_2 - a_1 X_1 - a_2 X_3)^2 \right] \]
    The normal equations are
    \[ \cov(X_2, X_1) = a_1 \cov(X_1, X_1) + a_2 \cov(X_3, X_1) \]
    \[ \cov(X_2, X_3) = a_1 \cov(X_1, X_3) + a_2 \cov(X_3, X_3) \]
    In matrix form
    \[ \begin{pmatrix} \gamma(0) & \gamma(2) \\ \gamma(2) & \gamma(0) \end{pmatrix} \begin{pmatrix} a_1 \\ a_2 \end{pmatrix} = \begin{pmatrix} \gamma(1) \\ \gamma(1) \end{pmatrix} \]
    Numerically 
    \[ R = \begin{pmatrix} 16/15 & 1/15 \\ 1/15 & 16/15 \end{pmatrix}, \quad v = \begin{pmatrix} -4/15 \\ -4/15 \end{pmatrix} \]
    We compute $\det R$ and $R^{-1}$
    
     \[   \det R = \left(\frac{16}{15}\right)^2 - \left( \frac1{15}\right)^2 = \frac{17}{15} \]
       \[ R^{-1} = \frac1{\det R} \begin{pmatrix} 16/15 & -1/15 \\ -1/15 & 16/15 \end{pmatrix} = \begin{pmatrix} 16/17 & -1/17 \\ -1/17 & 16/17 \end{pmatrix}\]
    Hence
    \[ \begin{pmatrix} a_1 \\ a_2 \end{pmatrix} = R^{-1} v = \begin{pmatrix} -4/17 \\ -4/17 \end{pmatrix} \]
    In other words
    \[ a_1 = a_2 = -\frac{4}{17},\quad \hat X_2 = -\frac{4}{17} (X_1+X_3) \]

    A standard formula for the MSE is 
    \[ \min \EE [(X_2 - \hat X_2^2)] = \gamma(0) - v^T R^{-1} v = \gamma(0) - v^T a \]
    We already have $\gamma(0) = 16/15$ and
    \[ v^T a = \begin{pmatrix} -4/15 & -4/15 \end{pmatrix} \begin{pmatrix} -4/17 \\ -4/17 \end{pmatrix} = \frac{32}{255} \]
    Thus
    \[ \min \EE[(X_2 - \hat X_2^2)] = \frac{16}{17} \] \qed

    \subsection*{Task 2}
    We compute autocovariances
    \begin{align*}
        \gamma(0) = \var(X_t) = \var(Z_t) + \theta^2 \var(Z_{t-1}) = \sigma^2 (1+\theta^2) \\
        \gamma(1) = \cov(X_t, X_{t-1}) = \cov(Z_t + \theta Z_{t-1}, Z_{t-1} + \theta Z_{t-2}) = \sigma^2 \theta
    \end{align*}
    and $\gamma(h) = 0$ for $|h| \ge 2$. The lag-1 correlation
    \[ \rho(1) = \frac{\gamma(1)}{\gamma(0)} = \frac{\theta}{1+\theta^2} \]
    Maximize $\rho(1)$ over $\theta$
    \[ \frac{\mathrm d}{\mathrm d\theta} \left[ \frac{\theta}{1+\theta^2} \right] = \frac{(1+\theta^2) - 2\theta^2}{(1+\theta^2)^2} = \frac{1-\theta^2}{(1+\theta^2)^2} = 0 \implies \theta^2 = 1 \implies \theta = \pm 1 \]
    Of these two, $\rho(1) = \theta/(1+\theta^2)$ is largest when $\theta = +1$. Hence
    \[ \theta^* = 1, \quad \rho(1)\bigg|_{\theta=1} = \frac12 \]

    For any MA(1), the spectral density is
    \[ f_X(\omega) = \frac{\sigma^2}{2\pi} \left| 1+ \theta e^{-i\omega} \right|^2 = \frac{\sigma^2}{2\pi} (1+\theta^2 - 2\theta \cos \omega) \]
    Plug in $\theta = 1$
    \[ f_X(\omega) = \frac{\sigma^2}{2\pi} (1+1+2 \cos\omega) = \frac{\sigma^2}{\pi} (1+\cos \omega) = \frac{2\sigma^2}{\pi} \cos^2 \left( \frac{\omega}{2} \right) \]
    $\cos^2 \left(\frac{\omega}2\right)$ is largest at $\omega = 0$ (value 1) and decreases to 0 at $\omega = \pi$. Thus at $\theta = 1$, low frequencies dominate the spectrum. As $\theta$ moves from 1:
    \begin{itemize}
        \item If $\theta$ decreases toward $0$, the spectrum flattens out (approaches white noise),
        \item If $\theta$ becomes negative, the $+2\theta \cos \omega$ term shifts power toward high frequencies (since $\cos \omega < 0$ near $\omega = \pi$).
    \end{itemize} \qed

    \subsection*{Task 3}
    In all cases we write
    \[ \phi(B) Y_t = \theta(B) Z_{t} \]
    with
    \[ \phi(B) = 1 - \phi_1 B - \phi_2 B^2 - \cdots, \quad \theta(B) = 1 + \theta_1 B + \theta_2 B^2 + \cdots \]
    An ARMA(1, 1) requires
    \begin{itemize}
        \item after cancelling any common factors, both $\phi(B)$ and $\theta(B)$ are of order at most 1,
        \item the single AR-root lies outside the unit circle (causality),
        \item the single MA-root lies outside the unit circle (invertibility).
    \end{itemize}

    \noindent \textbf{(i)}
    \[ Y_t - \frac56 Y_{t-1} = Z_t - \frac9{20} Z_{t-1} \]
    Here
    \[ \phi(B) = 1 - \frac56 B, \quad \theta(B) = 1 - \frac9{20} B \]
    Both are already degree 1, with no common factor to cancel. The AR-root solves $1 - \frac56 z = 0 \implies z = 6/5$, and $|6/5| > 1$, so the process is stationary/causal. The MA-root is $1-\frac{9}{20} z = 0 \implies z = 20/9, |20/9| > 1$, so it is invertible. Thus, (i) is causal, invertible ARMA(1, 1).

    \noindent \textbf{(ii)} \[ Y_t - Y_{t-1} + \frac14 Y_{t-2} = Z_t - \frac54 Z_{t-1} + \frac38 Z_{t-2} \] Here \[ \phi(B) = 1 - B + \frac14 B^2 = \left( 1 - \frac12 B\right)^2 \]
    \[ \theta(B) = 1 - \frac54 B + \frac38 B^2 = \left(1-\frac34 B\right)\left( 1-\frac12 B\right) \]
    There is a common factor $\left(1-\frac12 B\right)$. Cancel it to get
    \[ \phi^*(B) = 1 - \frac12 B, \quad \theta^*(B) = 1 - \frac34 B \]
    Now both are degree 1. The AR-root is $z = 2, |z| > 1$, so causal, and the MA-root is $z = \frac43, |z| > 1$, so invertible. Thus, (ii) reduces to causal, invertible ARMA(1, 1).

    \noindent \textbf{(iii)} \[ Y_t + \frac12 Y_{t-1} - \frac12 Y_{t-2} = Z_t - \frac54 Z_{t-1} + \frac38 Z_{t-2} \]
    Here
    \[ \phi(B) = 1 + \frac12 - \frac12 B^2 \implies \phi(z) = 1 + \frac12 z - \frac12 z^2 = 0 \implies z = 2 \text{ or } z = -1 \]
    Since one root has $|z| = 1$, the AR part fails causality (on the boundary), and there is no cancellation with the MA. One checks the MA factors as in (ii), but they share no common factor with $\phi(B)$. This it is a true ARMA(2, 2), non-causal. Thus, (iii) is not causal ARMA(1, 1).

    We take \[ Y_t - \frac56 Y_{t-1} = Z_t - \frac9{20} Z_{t-1} \]
    and choose the state vector
    \[ x_t = \begin{pmatrix} Y_t \\ Z_t \end{pmatrix}, \quad \text{input } u_t = Z_t, \quad \text{output } y_t = Y_t \]
    Then
    \[ \underbrace{\begin{pmatrix} Y_t \\ Z_t \end{pmatrix}}_{x_t} = \underbrace{\begin{pmatrix} 5/6 & -9/20 \\ 0 & 0 \end{pmatrix}}_{A} \begin{pmatrix} Y_{t-1} \\ Z_{t-1} \end{pmatrix} + \underbrace{\begin{pmatrix} 1 \\ 1\end{pmatrix}}_{B} Z_t \]
    and the observation equation is \[ y_t = \underbrace{\begin{pmatrix} 1 & 0 \end{pmatrix}}_{C} x_t \]
    Written in standard state-space form:
    \[
        \boxed{
            \begin{aligned}
                x_t &= A x_{t-1} + B u_t \\
                y_t &= C x_t
            \end{aligned}
        } \quad \text{where }
        A = \begin{pmatrix}
        5/6 & -9/20 \\
        0 & 0
        \end{pmatrix}, \quad
        B = \begin{pmatrix}
            1 \\ 1
        \end{pmatrix}, \quad
        C = \begin{pmatrix}
            1 & 0
        \end{pmatrix}
    \] 
    One checks immediately that the first row of the state equation reproduces $ Y_t = \frac56 Y_{t-1} - \frac9{20} Z_{t-1} + Z_t $ \qed

    \subsection*{Task 4}
    We are given
    \[ \begin{cases} X_t = b_t - (\theta_1 + \theta_2) Z_t - \theta_2 Z_{t-1} \\ b_t = b_{t-1} + (1+\theta_1 + \theta_2) Z_t \\ \{ Z_t \} \sim \mathrm{WN}(0, 1) \end{cases} \]
    Here $b_t$, is itself (unobserved) random walk driven by the same white noise.

    By definition,
    \[ \Delta X_t = X_t - X_{t-1} \]
    We substitute the expressions for $X_t$ and $X_{t-1}$:
    \begin{align*}
        \Delta X_t &= [b_t - (\theta_1 + \theta_2) Z_t - \theta_2 Z_{t-1}] - [ b_{t-1} - (\theta_1 + \theta_2) Z_{t-1} - \theta_2 Z_{t-2} ] \\
        &= (b_t - b_{t-1}) - (\theta_1 + \theta_2) (Z_t - Z_{t-1}) - \theta_2 (Z_{t-1} - Z_{t-2})
    \end{align*}
    From the second equation,
    \[ b_t - b_{t-1} = (1+\theta_1 + \theta_2) Z_t \]
    Hence
    \[ \Delta X_t = (1+\theta_1+\theta_2) Z_t - (\theta_1 + \theta_2) (Z_t-Z_{t-1}) - \theta_2 (Z_{t-1} - Z_{t-2}) \]
    Distribute the parantheses:
    \begin{align*}
        \Delta X_t &= (1+\theta_1 + \theta_2) Z_t - (\theta_1+\theta_2) Z_t + (\theta_1 + \theta_2) Z_{t-1} - \theta_2 Z_{t-1} + \theta_2 Z_{t-2} \\
        &= \underbrace{[(1+\theta_1 + \theta_2) - (\theta_1 + \theta_2)]}_{=1} Z_t + \underbrace{[(\theta_1 + \theta_2) - \theta_2]}_{=\theta_1} Z_{t-1} + \theta_2 Z_{t-2}
    \end{align*}
    Thus
    \[ \Delta X_t = Z_t + \theta_1 Z_{t-1} + \theta_2 Z_{t-2} \]
    which is exactly an MA(2) model. Since $\Delta X_t = X_t - X_{t-1}$ is MA(2), it follows that $\{X_t\}$ is integrated of order 1 with an MA(2) error -- that is,
    \[ X_t \sim \mathrm{ARIMA}(0,1,2) \] \qed

    \subsection*{Task 5}
    Recall the notation:
    \begin{itemize}
        \item $X_n = (X_1, \ldots, X_n)^T$,
        \item $\hat X_n = (\hat X_1, \ldots, \hat X_n)^T$,
        \item the one-step prediction errors
        $ \epsilon_t = X_t - \hat X_t$, 
        so $\epsilon_n = (\epsilon_1, \ldots, \epsilon_n)^T$,
        \item the prediction-error variances
        $ v_{t-1} = \EE[\epsilon_t^2]$,
        collected in the diagonal matrix
        \[ D_N = \mathrm{diag} (v_0, v_1, \ldots, v_{n-1}), \quad v_0 = \EE[\epsilon_1^2] = \var(X_1) \]
        \item the lower-triangular ,,Yule-Walker'' matrix:
        \[ \Phi_n = \begin{pmatrix} 1 & 0 & 0 & \ldots & 0 \\
            -\phi_{11} & 1 & 0 & \ldots & 0 \\
            -\phi_{21} & -\phi_{22} & 1 & \ldots & 0 \\
            \vdots & \vdots & \ddots & \ddots & \vdots \\
            -\phi_{n-1,1} & -\phi_{n-1,2} & \ldots & -\phi_{n-1, n-1} & 1 \end{pmatrix} \]
        which encodes the $t$-th-step prediction
        \[ \hat X_t = \sum_{j=1}^{t-1} \phi_{t-1, j} X_{t-j} \]
    \end{itemize}
    
    \noindent \textbf{(a)} Let $C_n = \Phi_n^{-1}$. Because the $\Phi_n$ is unit-lower-triangular, so is its inverse $C_n$, and in particular $(C_n)_{ii} = 1$. Next we observe that the defining relation
    \[ \Phi_n X_n = \epsilon_n \quad \implies \quad X_n = C_n \epsilon_n \]
    Since $\EE[\epsilon_n \epsilon_n^T] = D_n$, it follows immediately that
    \[ \Gamma_n = \EE[X_n X_n^T] = \EE[C_n \epsilon_n \epsilon_n^T C_n^T] = C_n D_n C_n^T \]
    as claimed.

    \noindent \textbf{(b)} By definition $\epsilon_n = X_n - \hat X_n$. Hence,
    \[ \hat X_n = X_n - \epsilon_n = C_n \epsilon_n - \epsilon_n = (C_n - I_n) \epsilon_n \]
    Equivalently,
    \[ \hat X_n = (C_n - I) \epsilon_n \]

    \noindent \textbf{(c)} We have
    \[ \gamma(0) = \var(X_t) = 1 + 1.2^2 = 2.44, \quad \gamma(1) = \cov(X_t, X_{t-1}) = -1.2, \quad \gamma(h) = 0 \ (h \ge 2) \]
    Hence the $t$-th-step prediction coefficients $\{\phi_{t-1,j}\}$ come from solving
    \[ \underbrace{\begin{pmatrix} \gamma(0) & \gamma(1) & \ldots \\ \gamma(1) & \gamma(0) & \gamma(1) \\ \vdots & \ddots & \ddots \end{pmatrix}}_{R} \begin{pmatrix} \phi_{t-1,1} \\ \phi_{t-1,2} \\ \vdots \end{pmatrix} = \begin{pmatrix} \gamma(1) \\ \gamma(2) \\ \vdots \end{pmatrix} \]
    
    By \emph{(b)},
    \[ \hat X_4 = (C_4 - I) \epsilon_4 = \begin{pmatrix} 0 & 0 & 0 & 0 \\ \phi_{11} & 0 & 0 & 0 \\ \phi_{21} & \phi_{22} & 0 & 0 \\ \phi_{31} & \phi_{32} & \phi_{33} & 0 \end{pmatrix} \begin{pmatrix} \epsilon_1 \\ \epsilon_2 \\ \epsilon_3 \\ \epsilon_4 \end{pmatrix} \]
    In particular
    \[ \hat X_4 = \phi_{31} \epsilon_1 + \phi_{32} \epsilon_2 + \phi_{33} \epsilon_3 \]
    Numerically (e.g. in R, see R-snippet~\ref{lst:phicoeff}) one finds
    \[ \phi_{31} \approx -0.723, \quad \phi_{32} \approx -0.469, \quad \phi_{33} \approx -0.230 \]
    so \[ \hat X_4 \approx -0.723 \epsilon_1 - 0.469 \epsilon_2 - 0.230 \epsilon_3 \]

    \begin{itemize}
        \item For $t = 2$, $\epsilon_2 = X_2 - \phi_{11} X_1$ with $\phi_{11} = \gamma(1) / \gamma(0) = -1.2 / 2.44 \approx -0.4918$. Hence \[ v_1 = \var(\epsilon_2) = \gamma(0) \left(1-\phi_{11}^2\right) \approx 1.8498 \]
        \item For $t = 3$, one solves the 2x2 normal equations to get
        $\phi_{22} \approx -0.6487, \phi_{22} \approx -0.3190$. Then
        \[ v_2 = \var(X_3 - \phi_{21} X_2 - \phi_{22} X_1) = \gamma(0) - (\gamma(1),\gamma(2)) \begin{pmatrix} \phi_{21} \\ \phi_{22} \end{pmatrix} \approx 1.6616 \]
    \end{itemize}

    \begin{lstlisting}[style=Rstyle,
                   caption={Solve for \(\phi_{21},\phi_{22}\) and \(\phi_{31},\phi_{32},\phi_{33}\) in R.},
                   label=lst:phicoeff]
# MA(1) parameters
theta <- -1.2

# autocovariances
gamma0 <- 1 + theta^2    # 2.44
gamma1 <- theta          # -1.2
gamma2 <- 0              # beyond lag 1

# (ii) 2x2 for phi21,phi22
R2 <- matrix(c(gamma0, gamma1,
               gamma1, gamma0), 2, 2)
r2 <- c(gamma1, gamma2)
phi2 <- solve(R2, r2)
phi21 <- phi2[1];  phi22 <- phi2[2]

# (i) 3x3 for phi31,phi32,phi33
R3 <- matrix(c(gamma0, gamma1, gamma2,
               gamma1, gamma0, gamma1,
               gamma2, gamma1, gamma0), 3, 3)
r3 <- c(gamma1, gamma2, 0)
phi3 <- solve(R3, r3)
phi31 <- phi3[1];  phi32 <- phi3[2];  phi33 <- phi3[3]

cat("phi21=",phi21,"\nphi22=",phi22,
    "\nphi31=",phi31,"\nphi32=",phi32,"\nphi33=",phi33,"\n")
\end{lstlisting} \qed

    \subsection*{Task 6}
    First observe that ,,invertible MA($q$)'' means exactly that the MA-polynomial
    \[ \theta(B) = 1 + \theta_1 B + \cdots + \theta_q B^q \]
    has all it's zeros \emph{outside} the unit circle. Equivalently there is a unique absolutely-summable sequence $\{\theta_j\}_{j=1}^\infty$ (with $\theta_j = 0$ form $j > q$) such that the infinite-past linear predictor of $X_{t+1}$ based on $\{X_t, X_{t-1}, \ldots \}$ is \[ \hat X_{t+1 \mid \infty} = \sum_{j=1}^\infty \theta_j (X_{t+1-j} - \hat X_{t+1-j}) \]
    A standard result in linear prediction theory then says
    \[ X_{t+1} - \hat X_{t+1 \mid \infty} = Z_{t+1} \]
    exactly, and that $\sum_{j=1}^\infty |\theta_j| < \infty$.

    By definition our one-step estimator based on the first $t$ observations is
    \[ \hat X_{t+1} = \sum_{j=1}^t \hat \theta_{t,j} (X_{t+1-j} - \hat X_{t+1-j}) \]
    where $\hat \theta_{t,j}$ solve the $t\times t$ Yule-Walker system. A key theorem is
    \begin{theorem}[Wold-Levinson]
        For an invertible process, as $t\to \infty$, \[ \hat \theta_{t,j} \to \theta_j \quad \text{for each fixed } j \]
        and in fact $\sum_{j=1}^\infty |\theta_j| < \infty$.
    \end{theorem}
    Therefore we can write the ,,prediction error minus the true innovation'' as
    \begin{align*}
        X_{t+1} - \hat X_{t+1} - Z_{t+1} &= \left( X_{t+1} - \hat X_{t+1\mid \infty} \right) - \left( \hat X_{t+1} - \hat X_{t+1 \mid \infty} \right) - Z_{t+1} \\
        &= Z_{t+1} - Z_{t+1} - \sum_{j=1}^\infty \theta_j (X_{t+1-j} - \hat X_{t+1-j}) + \sum_{j=1}^t \hat \theta_{t,j} (X_{t+1-j} - \hat X_{t+1-j}) \\
        &= \sum_{j=1}^t (\hat \theta_{t,j} - \theta_j) \Delta_{t+1-j} - \sum_{j=t+1}^\infty \theta_j \Delta_{t+1-j}
    \end{align*}
    where we have written $\Delta_s = X_s - \hat X_s$. Because
    \begin{enumerate}
        \item $\Delta_s$ is uniformly $L^2$-bounded (it's just a linear combination of finitely many $Z$'s),
        \item $\hat \theta_{t,j} \to \theta_j$ for each fixed $j$, and
        \item $\sum_{j=1}^\infty |\theta_j| < \infty$
    \end{enumerate}
    one shows by dominated convergence (or the Cauchy criterion) that the right-hand side converges to zero in mean square. Hence
    \[ \EE[(X_{t_1} - \hat X_{t+1} - Z_{t+1})^2] \rightarrow 0 \]
    which is exactly part (a).

    By definition \[ v_t = \EE[(X_{t+1} - \hat X_{t+1})^2] \]
    Now \[ X_{t+1} - \hat X_{t+1} = (Z_{t+1}) + \underbrace{[X_{t+1} - \hat X_{t+1} - Z_{t+1}]}_{\to 0 \text{ in } L^2} \]
    Call the bracketed term $R_{t+1}$. Then
    \[ v_t = \EE[(Z_{t+1} + R_{t+1})^2] = \EE[Z_{t+1}^2] + \EE[{R_{t+1}^2}] + 2 \EE[Z_{t+1} R_{t+1}] \]
    But $R_{t+1}$ is a (mean-zero) function of $\{ Z_s \colon s \le t \}$, so it is uncorrelated with $Z_{t+1}$. Hence the cross-term vanishes, and \[ v_t = \sigma^2 + \EE[R_{t+1}^2] \rightarrow \sigma^2 + 0 = \sigma^2 \]
    That completes part (b).

    As we observe more and more of the past, our finite-sample predictor $\{\hat X_{t+1} \}$ converges for the ,,ideal'' infinite-past best linear predictor (the Wold predictor), whose one-step-ahead error is exactly the current innovation $Z_{t+1}$. Thus eventually our prediction error variance converges to $\var(Z_{t+1}) = \sigma^2$. \qed

\end{document}